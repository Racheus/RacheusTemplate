\documentclass[xcolor=x11names,compress,aspectratio=169]{ctexbeamer}
\usetheme{Darmstadt} 
\useoutertheme[subsection=false,footline=authortitle]{miniframes}
\setbeamertemplate{navigation symbols}{} 
\logo{\includegraphics[height=0.07\textwidth]{Figure/sjtu-red.png}}
\setbeamertemplate{background}{\includegraphics[width=\paperwidth]{Figure/sjtu-miao.png}}
\usepackage{datetime}
\newdateformat{mydate}{\monthname[\THEMONTH] \THEDAY, \THEYEAR}
%\usefonttheme[onlymath]{serif}


\usepackage{fontspec}
\usefonttheme{serif}
\input{format.tex}
\usepackage{listings}

\lstset{
    language=C++, % 设置语言为C++
    basicstyle=\ttfamily\small, % 设置代码字体样式
    numbers=left, % 在左侧显示行号
    numberstyle=\tiny\color{gray}, % 行号样式
    keywordstyle=\color{blue}, % 关键字颜色
    commentstyle=\color{green!60!black}, % 注释颜色
    stringstyle=\color{red}, % 字符串颜色
    breaklines=true, % 自动换行
    columns=flexible,%不随便添加空格,只在已经有空格的地方添加空格,
%如果想要添加空格使用fixed作为参数(这是默认的),如果坚决不添加空格使用fullflexible作为参数
    frame=single, % 显示边框
    backgroundcolor=\color{lightgray!10}, % 背景色
}



\title{ Graduation Project  (ME4918)\\
  \textbf{Weekly Report(X)}}

\author{\selectfont\kaishu 赵四维}
 
\institute
{
   Institute of Robotics,\\
   School of Mechanical Engineering,SJTU 
}

\date{\mydate\today}


\begin{document}

\maketitle

\AtBeginSection[]
{
  \begin{frame}
    \frametitle{Table of Contents}
    \tableofcontents[currentsection,hideallsubsections]
  \end{frame}
}

\section{Figure and Notes}
\subsection{Figure and Notes}

\begin{frame}{Figure and Notes}
In Simulink, you can import models from other modeling environments, such as Solidworks.
\begin{figure}
\includegraphics[width=0.8\textwidth]{Figure/logo.png}
\end{figure}

Save the modeling file in XML format or URDF format, set a reasonable rotation coordinate system, and import it into Simulink for simulation.
\end{frame}

\section{Chinese Characters}
\subsection{中文测试}
\begin{frame}{中文测试}
\vspace{-2cm}
\hspace{3cm}
\begin{itemize}
  \item 邓紫棋
  \item 林俊杰
\end{itemize}

\begin{center}
  \textbf{手心的蔷薇(加粗)}

  \selectfont \kaishu 刺伤而不自觉(楷书)
  
  \selectfont \heiti 你值得被疼爱(黑体)
  
  \selectfont \fangsong 你懂我的期待(仿宋)
\end{center}

\begin{block}{代表作展示}
  \begin{itemize}
    \item 邓紫棋:《倒数》,《多远都要在一起》
    \item 林俊杰:《修炼爱情》,《可惜没如果》
  \end{itemize}
\end{block}

\end{frame}  

\section{Mathematical Formula}
\subsection{Mathematical Formula}
\begin{frame}{Mathematical Formula}
\begin{block}{Mathematical Formula}
  \begin{equation}
    \begin{aligned}
      \dot{x} &= Ax + Bu \\
      y &= Cx + Du
    \end{aligned}
  \end{equation}
\end{block}

对于机械臂的拉格朗日动力学方程,可以写成如下形式:
\begin{equation}
  \begin{aligned}
    \frac{d}{dt}(\frac{\partial L}{\partial \dot{q}_i}) - \frac{\partial L}{\partial q_i} &= Q_i
  \end{aligned}
\end{equation}

Where Lagrangian $L$ is defined as:
\begin{equation}
  \begin{aligned}
    L &= T - V
  \end{aligned}
\end{equation}
\end{frame}

\section{Coding Environment}
\subsection{Coding Environment}
\begin{frame}{Coding Environment}
import numpy as np

\begin{figure}
\includegraphics[width=0.8\textwidth]{Figure/code.png}
\end{figure}

CodeSnap + Figure 清晰度不太高


\end{frame}


\begin{frame}[fragile] % 使用fragile选项以在frame内使用lstlisting环境
  \frametitle{lstlisting Environment}
  
  Here is a sample C++ code snippet:\\
  
  \begin{lstlisting}[caption={Sample C++ Code}]
  #include <iostream>
  
  int main() {
      std::cout << "Hello, Beamer!" << std::endl;
      return 0; //test
  }
  \end{lstlisting}
  
  \end{frame}

  
  \begin{frame}
    \begin{center}
      \vspace{1cm}
      \Huge \textbf{Thanks for listening!}
      \vspace{2cm}
      \includegraphics[width=0.8\textwidth]{Figure/siyuan.jpg} % Optional: Add an image
    \end{center}
  \end{frame}
\end{document}